\documentclass{article}

\usepackage[margin=1in]{geometry}
\usepackage{amsmath}
\usepackage{spverbatim}
\usepackage{graphicx}
\usepackage{multicol}

\begin{document}
	
\title{ESOF 322 - Homework 5}
\author{Nathan Stouffer and Kevin Browder}

\maketitle
\newpage

\section*{Question 1}
\subsection*{Part A}
Problem: Create a control flowgraph for the sieve algorithm. To the left of the line numbers in the
source code clearly identify the nodes that will be used in your graph. Once you have
identified the nodes, draw the control graph. \\\\
Solution: Drawn below next to the C code.
\begin{spverbatim}
	1. /* Find all primes from 2-upper_bound using Sieve of Eratosthanes */
	2.
	3. #include
	4. typedef struct IntList {
	5.     int value;
	6.     struct IntList *next;
	7.     } *INTLIST, INTCELL;
	8. INTLIST sieve (int upper_bound ) {
	9.
	10.     INTLIST prime_list = NULL; /* list of primes found */
	11.     INTLIST cursor; /* cursor into prime list */
	12.     int candidate; /* a candidate prime number */
	13.     int is_prime; /* flag: 1=prime, 0=not prime */
	14.
	15.     /* try all numbers up to upper_bound */
	16.     for (candidate=2;
	17.
	18.         candidate <= upper_bound;
	19.         candidate++) {
	20.
	21.             is_prime = 1; /* assume candidate is prime */
	22.             for(cursor = prime_list;
	23.
	24.                 cursor;
	25.                 cursor = cursor->next) {
	26.
	27.                     if (candidate % cursor->value == 0) {
	28.
	29.                         /* candidate divisible by prime */
	30.                         /* in list, can't be prime */
	31.                         is_prime = 0;
	32.                         break; /* "for cursor" loop */
	33.                     }
	34.             }
	35.             if(is_prime) {
	36.
	37.                 /* add candidate to front of list */
	38.                 cursor = (INTLIST) malloc(sizeof(INTCELL));
	39.                 cursor->value = candidate;
	40.                 cursor->next = prime_list;
	41.                 prime_list = cursor;
	42.             }
	43.         }
	44.     return prime_list;
	45. }
\end{spverbatim}

\subsection*{Part B}
Problem: Provide a set of test cases that would give 100\% Node Coverage (NC). \\\\
Solution:\\
Test Case 1: \\
Input: 4 \\
Expected Output: [3, 2]\\
Actual Output: [3, 2]\\
$T = \{ t_1 = \{1, 2, 3, 4, 5, 6, 7, 8, 9, 10, 11, 12, 13, 14\}\}$
\subsection*{Part C}
Problem: Provide a set of test cases that would give 100\% Edge Coverage (EC). \\\\
Solution: \\
Test Case 1: \\
Input: 4 \\
Expected Output: [3, 2]\\
Actual Output: [3, 2]\\
$T = \{ t_1 = \{1, 2, 3, 4, 5, 6, 7, 8, 9, 10, 11, 12, 13, 14\}\}$
\subsection*{Part D}
Problem: Is 100\% NC or 100\% EC possible in general? Why, or why not? \\\\
Solution: \\

Generally it is possible because there a finite number of nodes and edges and you simply must come up with test cases to cover all of them.
\newpage

\section*{Question 2}

\subsection*{Part A}
Problem: Draw the execution of the calls that exhibit the YoYo problem for a runtime trace of
C3.B, and for C4.A (Draw both on the diagram below). \\\\
\begin{figure}[h]
	\centering
	\includegraphics[width=6in]{yo-yo-diagram.png}
\end{figure}

\subsection*{Part B}
Problem: Describe what happens when we call C1.D \\\\
Solution: 

\newpage

\section*{Question 3}
Given the following program:
\begin{spverbatim}
	1. public int fibonacci (int i) { 
	2.     int fib1 = 1;    // fib(n-1)
	3.     int fib2 = 1;    // fib(n-2)
	4.     int fib = 0;
	5.     int j;

	6.     if (i <= 1) 
	7.         fib = 1;
	       else 
	8.        for (j = 1;
	9.             j < i;
	10.             j++) {
	11.            fib = fib2 + fib1;
	12.            fib2 = fib1;
	13.            fib1 = fib;
	          }
	14.    return fib;		
	   }
\end{spverbatim} \bigskip
\noindent
Problem: Give test cases that will kill the following mutations: \\
a. Line 6: if (i $<$ 1)	\\
b. Line 6: if (i == 1)	\\
c. Line 12: fib2 = fib;	\\\\
Solution:\\
\begin{multicols}{2}
\noindent
a. normal: \\
	input = 1 \\
	return = 1 \\
	expected = 1 \\
	mutant: \\
	input = 1 \\
	return = 0 \\
	expected = 1\\
Returns not equal, mutant killed \\\\\\
b.normal: \\
input = 0 \\
return = 1 \\
expected = 1 \\
mutant: \\
input = 0 \\
return = 0 \\
expected = 1 \\
Returns not equal, mutant killed \\\\
\end{multicols}
\noindent
c.normal: \\
input = 3 \\
return = 3 \\
expected = 3 \\
mutant: \\
input = 3 \\
return = 4 \\
expected = 3 \\
Returns not equal, mutant killed \\\\

	

\end{document}